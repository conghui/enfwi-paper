% gjilguid2e.tex
% V2.0 released 1998 December 18
% V2.1 released 2003 October 7 -- Gregor Hutton, updated the web address for the style files.

\documentclass[mreferee]{gji}
\usepackage{timet}
\usepackage{amssymb}
\usepackage[cmex10]{amsmath}
\usepackage{mathtools, cuted}

\title[Geophys.\ J.\ Int.: \LaTeXe\ Guide for Authors] 
	{Ensemble Source Encoding Full-waveform Inversion with Self-Adaptive Regularization}
	
%\author[B.L.N. Kennett]
%  {B.L.N. Kennett$^1$\thanks{Pacific Region Office, GJI} \\
%  $^1$ Research School of Earth Sciences, Australian National
%    University, Canberra ACT \emph{0200}, Australia
%  }
  
\author[C. He]
  {Conghui He$^{12}$, Yushu Chen$^2$, Bingwei Chen$^{12}$, Haohuan Fu$^{123}$ \\
  $^1$ Ministry of Education Key Laboratory for Earth System Modeling, Department of Earth System Science, Tsinghua University, Beijing, China \\
  $^2$ Department of Computer Science and Technology, Tsinghua University, Beijing, China \\  
  $^3$ Laboratory for Regional Oceanography and Numerical Modeling, Qingdao National Laboratory for Marine Science and Technology
  }
    
\date{Received 1998 December 18; in original form 1998 November 22}
\pagerange{\pageref{firstpage}--\pageref{lastpage}}
\volume{200}
\pubyear{1998}

%\def\LaTeX{L\kern-.36em\raise.3ex\hbox{{\small A}}\kern-.15em
%    T\kern-.1667em\lower.7ex\hbox{E}\kern-.125emX}
%\def\LATeX{L\kern-.36em\raise.3ex\hbox{{\Large A}}\kern-.15em
%    T\kern-.1667em\lower.7ex\hbox{E}\kern-.125emX}
% Authors with AMS fonts and mssymb.tex can comment out the following
% line to get the correct symbol for Geophysical Journal International.
\let\leqslant=\leq

\newtheorem{theorem}{Theorem}[section]

\begin{document}

\label{firstpage}

\maketitle


\begin{summary}
We develop an ensemble source encoding full-waveform inversion method with self-adaptive regularization (EnFWI), which approximates the total inversion using the ensemble Kalman filter (EnKF) to alleviate the error caused by inaccurate initial model and data noise with high computational efficiency. It refines an ensemble of velocity models simutaneously by iterative incorporating the waveform observation, and approximate the nonlinear evolution of error covariance by the ensemble covariance. To overcome the problem in EnKF that the updating direcitons are confined in the low rank space spanned by the model samples which is lack of representation ability in general, we apply the encoded simultaneous-source FWI (ESSFWI) to refine each model sample, so that the gradient directions with perturbation are introduced to improve the representation ability of the low rank updating space. Along with the velocity, we use the EnKF scheme to adjust the anisotropic regularization factors in order to utilize the smoothing constrains in different directions to further reduce the model error. Experiments show that EnFWI achieves larger convergence range and better tolerance to data noise with less computational costs than traditional FWI method.
\end{summary}

\begin{keywords}
Inverse theory; Controlled source seismology; Computational seismology; Wave propagation.
\end{keywords}

\section{Introduction}

Full-waveform inversion (FWI) is a promising data-fitting procedure which aims to extract quantitative information of the earth for high-resolution imaging from seismograms \cite{ta84}. However, FWI is a non-linear ill-posed problem which suffers from local convergence led by the initial model error and the lack of ultra-low frequency (ULF) data \cite{vi}. Data noise further deteriorates the imaging results. 

Traditionally, the initial model is provided by some other methods such as traveltime tomography \cite{ls,pg,ss,pc,op,wo,wa} and migration velocity analysis \cite{al,sc,sb}. Other approaches to generate the initial model include first-arrival traveltime tomography \cite{ho}, stereotomography \cite{la}, Laplace-domain inversion and Laplace-Fourier-domain inversion \cite{sc08,sc09}, seismic envelope inversion \cite{wu,lw} and so on. These methods recover long-wavelength background structure to a certain extent, obtaining initial models with relatively low error. However, it is difficult to meet the demands of conventional gradient-based FWI that the initial model should produce synthetic data that are within a half-cycle, everywhere \cite{al}. Consequently, inversion methods with larger convergence domain are required. 

Multiscale FWI methods \cite{bu,ra,vigh} are developed to reduce the initial model dependence by performing a cascade inversion starting from the lowest frequency available. The lowest frequency band (1.5-2.0 Hz) is crucial for these methods to recover the correct long-wavelength background-velocity structure. However, in most cases, the standard seismic records can go down only to approximately 5 Hz \cite{wu}. Not enough low frequency information is contained in the seismograms to avoid cycle skipping.

The conventional FWI is a simplification of the total inversion \cite{ta82}. Instead of minimizing the data misfit directly, the total inversion maximizes the posterior probability of the model parameters under certain initial model and seismograms as measurement. Comparing to FWI, the total inversion is less sensitive to initial model error and data noise \cite{ta84}. However, in the total inversion, the model and data covariance are required to generate the probability density function, leading to huge computation and memory costs, which make it actually impossible for high dimensional seismic inversion problems even with the state of art supercomputers.

The ensemble Kalman filter (EnKF) incorporates the initial model state and the observation using the same objective function as the total inversion \cite{ev94,ev03,ev09}. Instead of generating the covariance explicitly, it keeps an ensemble of model parameters, and uses the ensemble covariance to approximate the real covariance, providing an efficient mean to alleviate the computational costs. It has been applied in seismic inversion \cite{ji,qh}. However, in EnKF, the updating increment is confined in the low rank space spanned by the model samples. In high-dimensional problems, e.g., seismic explorations, where model parameters are several order of magnitudes larger than model samples, the low rank updating space is usually difficult to provide significant decreasing of the data misfit when the initial ensemble samples are generated by random perturbation. It is critical to adjust the model samples to improve the representation ability of the low rank space they spanned.

FWI with encoded simultaneous sources (ESSFWI) is an efficient tool to obtain a useful updating direction \cite{kr,be,ca}. It stacks the seismograms to reduce the number of forward modeling, which significantly reduced the computation, and use different source codes in each updating step to suppress the noise resulting from crosstalk between the encoded sources. By the small fluctuations caused by source encoding, ESSFWI may even overcome small local minima \cite{ca}. However, ESSFWI is more sensitive to data noise \cite{kr} than FWI.

In this paper, we propose an ensemble source encoding full-waveform inversion method with self-adaptive regularization (EnFWI). In this method, we use EnKF to approximate the total inversion to enlarge the convergence domain and promote the resistance to data noise. ESSFWI is applied to refine the model samples to improve the representation for the low rank approximation. We also utilize the EnKF scheme to adjust the anisotropic regularization factor along with the velocity model, in order to utilize the smoothing constrains in different directions to further reduce the model error.

This paper is organized as follows. We firstly derive our EnFWI method. Then, we apply the method in the Marmousi model, to investigate the convergence range, the tolerance of data noise and computational costs of EnFWI.


\section{Method}

In this section, we design the EnFWI scheme. We first apply the EnKF in seismic inversion problems to approximate the total inversion with low computational cost. Then we introduce ESSFWI to refine the model samples in order to improve the representation ability of the low rank updating space. We also use the EnKF scheme to adjust the anisotropic regularization factors in ESSFWI along with the model samples.

\subsection{Approximating the total inversion by EnKF}

In the seismic inversion problem, we need to obtain the underground model, from an initial model and the seismogram, which serves as an observation of the wave field. We here follow \cite{ev03} and apply the general EnFK theory to the seismic inversion problem.
For simplicity, we consider the acoustic FWI problem. The scalar wave equation is

\begin{equation}
\frac{m(x)\partial ^2{u}(x,t)}{\partial{t^2}}-\Delta u(x,t)=f
\end{equation}
where $u$ is the pressure; the slowness $m=1/c$ is the model parameter to solve; $c$ is the wave velocity, and $f$ is the source term. The seismogram $d$ is the pressure $u$ recorded in seismic surveys. We define the measurement operator $H$ as $H_m=d_{cal}(m)$, where $d_{cal}(m)$ is the synthetic seismogram.

The total inversion and EnKF solve the inversion problem within the Bayesian framework. Assuming the posterior probability density function (PDF) to be Gaussian, in the $k$th updating step, by combining the prior knowledge of the initial model $(m_b)_k$, which is called the background field, and the observation $d_k$, the posterior PDF $p$ satisfies


\begin{equation}
%\resizebox{0.5\textwidth}{!}{$p \varpropto \exp\left ( -\left ( \left ( \left ( m_b \right )_k - m \right )^T B^{-1} \left ( \left ( m_b \right )_k -m\right ) +\left ( d_k-H_m \right )^TR_k^{-1}\left ( d_k-H_m \right )\right ) \right )$}
p \varpropto \exp\left ( -\left ( \left ( \left ( m_b \right )_k - m \right )^T B^{-1} \left ( \left ( m_b \right )_k -m\right ) +\left ( d_k-H_m \right )^TR_k^{-1}\left ( d_k-H_m \right )\right ) \right )
\end{equation}
where $B$ is the covariance of $m_0$ and $R_k$ is the covariance of $d_k$. 

To obtain the optimal model $m$ to maximize the posterior PDF $p$, we minimize the objective function


\begin{acknowledgments}
A number of colleagues have helped with suggestions for the
improvement of this material and I would particularly like to thank
Bob Geller, University of Tokyo for his criticisms and corrections.
\end{acknowledgments}

\begin{thebibliography}{}
  \bibitem[\protect\citename{Alkhalifah \& Choi }2014]{al} Alkhalifah, T. \& Choi, Y., 2014. \textit{From tomography to full-waveform inversion with a single objective function}, Geophysics, 79, R55-R61.
  \bibitem[\protect\citename{Ben-Hadj-Ali et al. }2011]{be} Ben-Hadj-Ali, H., Operto, S. \& Virieux, J., 2011. \textit{An efficient frequency-domain full waveform inversion method using simultaneous encoded sources}, Geophysics, 76, R109-R124.
  \bibitem[\protect\citename{Castellanos et al. }2015]{ca}  Castellanos, C., Metivier, L., Operto, S., Brossier, R. \& Virieux, J., 2015. \textit{Fast full waveform inversion with source encoding and second-order optimization methods}, Geophysical Journal International, 200, 720-744. 
  \bibitem[\protect\citename{Evensen }1994]{ev94}  Evensen, G., 1994. \textit{Sequential data assimilation with a nonlinear quasi-geostrophic model using Monte Carlo methods to forecast error statistics}, Journal of Geophysical Research: Oceans, 99, 10143-10162.
  \bibitem[\protect\citename{Evensen }2003]{ev03} Evensen, G., 2003. \textit{The Ensemble Kalman Filter: theoretical formulation and practical implementation}, Ocean Dynamics, 53, 343-367.
  \bibitem[\protect\citename{Evensen }2009]{ev09} Evensen, G., 2009. \textit{The ensemble Kalman filter for combined state and parameter estimation}, Control Systems, IEEE, 29, 83-104.
  \bibitem[\protect\citename{Krebs et al. }2009]{kr} Krebs, J.R., Anderson, J.E., Hinkley, D., Neelamani, R., Lee, S., Baumstein, A. \& Lacasse, M.-D., 2009. \textit{Fast full-wavefield seismic inversion using encoded sources}, Geophysics, 74, WCC177-WCC188.
  \bibitem[\protect\citename{Luo \& Schuster }1991]{ls} Luo, Y. \& Schuster, G.T., 1991. \textit{Wave-equation traveltime inversion}, Geophysics, 56, 645-653.
\bibitem[\protect\citename{Luo \& Wu }2015]{lw} Luo, J. and Wu, R.-S., 2015. Seismic envelope inversion: reduction of local minima and noise resistance, Geophysical Prospecting, 63, 597-614.
\bibitem[\protect\citename{Tarantola \& Valette }1982]{ta82} Tarantola, A. \& Valette, B., 1982. Generalized nonlinear inverse problems solved using the least squares criterion, Rev Geophys, 20, 219-232.
\bibitem[\protect\citename{Woodward et al. }2008]{wo} Woodward, M.J., Nichols, D., Zdraveva, O., Whitfield, P. \& Johns, T., 2008. \textit{A decade of tomography}, GEOPHYSICS, 73, VE5-VE11.
\bibitem[\protect\citename{Wu et al. }2014]{wu} Wu, R.-S., Luo, J. \& Wu, B., 2014. Seismic envelope inversion and modulation signal model, Geophysics, 79, WA13-WA24.
\bibitem[\protect\citename{Al-Yahya }1989]{al} Al-Yahya, K., 1989. \textit{Velocity analysis by iterative profile migration}, Geophysics, 54, 718?729.
\bibitem[\protect\citename{Hole }1992]{ho} Hole, J. A., 1992. Nonlinear high-resolution three-dimensional seismic travel time tomography, Journal of Geophysical Research, 97, 6553?6562.
\bibitem[\protect\citename{Bunks et al. }1995]{bu} Bunks, C., F.M. Saleck, S. Zaleski, and G. Chavent, 1995. Multiscale seismic waveform inversion,  Geophysics, 60, 1457?1473.
\bibitem[\protect\citename{Lambar$\acute{\textrm{e}}$ }2008]{la} Lambar$\acute{\textrm{e}}$, G., 2008. Stereotomography: Geophysics, 73, no. 5, VE25?VE34.
\bibitem[\protect\citename{Jin et al. }2008]{ji} Jin, L., Sen, M.K. \& Stoffa, P.L., 2008. One-dimensional prestack seismic waveform inversion using ensemble Kalman filter, SEG Annu. Meet. Las Vegas, USA.
%Kalman, E., 1960. A new approach to linear filtering and prediction problems. Journal of Fluids Engineering, 82(1), 35-45.
\bibitem[\protect\citename{Operto et al. }2004]{op} Operto, S., C. Ravaut, L. Improta, J. Virieux, A. Herrero, \& P. Dellaversana, 2004. \textit{Quantitative imaging of complex structures from dense wide aperture seismic data by multiscale traveltime and waveform inversions: A case study}, Geophysical Prospecting, 52, 625-651.
\bibitem[\protect\citename{Pratt \& Goulty }1991]{pg} Pratt, R. G., \& N. R. Goulty, 1991. \textit{Combining wave-equation imaging with traveltime tomography to form high-resolution images from crosshole data}, Geophysics, 56, 208-224.
\bibitem[\protect\citename{Priolo \& Chiaruttini }2003]{pc} Priolo, E., \& C. Chiaruttini, 2003. \textit{Analytical and numerical analysis of tomographic resolution with band-limited signals}, Geophysics, 68, 600-613.
\bibitem[\protect\citename{Quan \& Harris }2008]{qh} Quan, Y. \& Harris, J.M., 2008. Stochastic seismic inversion using both waveformand traveltime data and its application to time-lapse monitoring, SEG Annu. Meet. Las Vegas, USA.
\bibitem[\protect\citename{Sava \& Biondi }2004]{sb} Sava, P., and B. Biondi, 2004. Wave-equation migration velocity analysis?Part I: Theory, \textit{Geophysical Prospecting}, 52, 593?606.
\bibitem[\protect\citename{Shipp \& Singh }2002]{ss} Shipp, R. M., \& S. C. Singh, 2002. \textit{Two-dimensional full wavefield inversion of wide-aperture marine seismic streamer data}, Geophysical Journal International, 151, 325-344.
\bibitem[\protect\citename{Shin \& Cha }2008]{sc08} Shin, C., and Y. H. Cha, 2008. Waveform inversion in the Laplace domain: Geophysical Journal International, 173, 922?931. 
\bibitem[\protect\citename{Ravaut et al. }2004]{ra} Ravaut, C., S. Operto, L. Improta, J. Virieux, A. Herrero, and P. Dell' Aversana, 2004. Multiscale imaging of complex structures from multifold wide-aperture seismic data by frequency-domain full-waveform tomography: Application to a thrust belt, Geophysical Journal International, 159, 1032?1056.
\bibitem[\protect\citename{Shin \& Cha }2009]{sc09} Shin, C., and Y. H. Cha, 2009. Waveform inversion in the Laplace-Fourier domain: Geophysical Journal International, 177, 1067?1079.
\bibitem[\protect\citename{Symes \& Carazzone }1991]{sc} Symes, W., and J. J. Carazzone, 1991. Velocity inversion by differential semblance optimization: Geophysics, 56, 654?663.
\bibitem[\protect\citename{Vigh \emph{et al.} }2011]{vigh} Vigh, D., J. Kapoor, N. Moldoveanu, and H. Li, 2011. Breakthrough acquisition and technologies for subsalt imaging, Geophysics, 76, no. 5, WB41?WB51.
\bibitem[\protect\citename{Wang et al. }2014]{wa} Wang, H., S. C. Singh, and H. Calandra, 2014. \textit{Integrated inversion using combined wave-equation tomography and full waveform inversion}, Geophysical Journal International, 198, 430?446.


  

  \bibitem[\protect\citename{Tarantola }1984]{ta84} Tarantola, A., 1984. \textit{Inversion of seismic reflection data in the acoustic approximation}, Geophysics, 49, 1259-1266.
  \bibitem[\protect\citename{Virieux \& Operto }2009]{vi}  Virieux, J. \& Operto, S., 2009. \textit{An overview of full-waveform inversion in exploration geophysics}, Geophysics, 74, WCC1-WCC26.

\end{thebibliography}



\end{document}
